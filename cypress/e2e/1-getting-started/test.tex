 [บรรทัดที่2] = it คือชื่อของ Test Case
 [บรรทัดที่3] = การใส่ URL
 [บรรทัดที่4] = ค้นหา id แบบที่ 1 โดยใส่ #
 [บรรทัดที่5] = ค้นหา id แบบที่ 2 โดยใส่ [id="title-container"]
 [บรรทัดที่6] = ค้นหา id แบบที่ 3 โดยมี .invoke('text').as('header-text') ต่อท้ายเอาไว้ ค้นหาข้อความ
 [บรรทัดที่7] = ค้นหา เช็คข้อความ **
 [บรรทัดที่8] = เช็คว่าเห็น h2 ไหม อันนี้เป็นแบบเห็น
 [บรรทัดที่9] = เช็คว่าเห็น h2 ไหม อันนี้เป็นแบบไม่เห็น
 [บรรทัดที่10] = กรอก Username
 [บรรทัดที่11] = กรอก Password
 [บรรทัดที่12] = เป็นปุ่มกด submit เพื่อ login
 [บรรทัดที่14] = จากที่ submit จากบรรทัดที่ 12 จะไปอีกลิ้งค์ที่ได้ใส่ไว้
 [บรรทัดที่15] = แสดงข้อความ Logged In Successfully
 [บรรทัดที่16] = มีปุ่ม Log out 


 it.only = สัง run ทีละ test case
//https://practicetestautomation.com/practice-test-login/  URL สำหรับทดสอบหน้า Login ตาม Test Case